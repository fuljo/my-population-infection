\section{Algorithm description}
\label{sec:alg_desc}

Here we give a high-level description of how the program works, with some pseudocode where useful. Such pseudocode is meant to provide an abstract and concise representation of the algorithm, leaving out many implementation details and optimizations.

\paragraph{Individual}
An individual is described by the following attributes:
\begin{itemize}
    \item $id$ : unique identifier in the world (assigned incrementally)
    \item $pos = (x,y)$ : absolute position in the world
    \item $\Delta pos = (\Delta x,\Delta y)$ : displacement vector, representing the movement at each step
    \item $status =$ \NotExposed | \Exposed | \Infected | \Immune : current status of the individual. Both \NotExposed and \Exposed individuals are considered susceptible; the distinction is introduced for computational needs, for ease of visualization and debugging
    \item $t_{status}$ : time passed since the individual entered its current state. It is used to determine when to change state.
\end{itemize}

\paragraph{Initialization}
The program starts by reading the simulation parameters from command line and validating them according to the rules we mentioned in section~\ref{sec:intro}; if an error is found, it is logged back to the user and the program exits with \MPIAbort.
The configuration is then broadcast to all the processes with \MPIBcast, using the purposefully created \texttt{mpi\_global\_config} datatype.
Each country proceeds to autonomously calculate its $x,y$ limits and the indices of its neighbors.

Now the root proceeds to (uniformly) distribute the population among the countries, sending the local number of individuals $N_c$ and infected $I_c$ via \MPIScatter. Each country will then create the requested number of individuals on its own, giving them a random position inside its borders and a random travelling direction.
We also use \MPIExscan to compute the id of the first individual of each country, by summing the numbers of individuals of the previous ones.

Note that we used blocking and asynchronous communication primitives for two main reasons:
\begin{enumerate*}[label=(\roman*)]
    \item the amount of data to be sent is small, so it is unlikely to fill the send-buffers, and
    \item the data flows only from the root to the other processes, so there is no need for synchronization
\end{enumerate*}.
Pseudocode of the initialization procedure can be found in algorithm~\ref{alg:initialization}.

\paragraph{Main loop}
Each iteration of the main loop takes a consistent situation at time $t$ and computes a consistent situation at $t + t_{step}$. There is no interpolation: each individual is treated as if it were in the same initial position and status for the whole step. At the beginning of each iteration we assume that each susceptible individual has $status_i = \NotExposed$.
The procedure, summarized in algorithm~\ref{alg:mainloop}, can be broken down in various phases:
\begin{description}
    \item[update exposure] Check each susceptible individual $i$ against infected individuals: as soon as $i$ is found within spreading distance $d$ from an infected $j$, it is flagged with $status_i = \Exposed$.

    \item[update status] Based on $status$ and $t_{status}$, the status of each individual is updated (for example, an infected individual with $t_{status} >= t_{infection}$ will become immune). In case the individual changed status or was not exposed, $t_{status}$ is reset to zero, otherwise it is incremented by $t_{step}$. \Exposed individuals are set back to \NotExposed. In this way $t_{status}$ will accumulate the time an individual has been continuously exposed.
     
    \item[update position] $pos_i$ is updated by summing the displacement $\Delta pos_i$. In case the resulting position is out of the country boundary, there are two possibilities:
    \begin{enumerate*}[label=(\roman*)]
        \item it is also out of the world boundary, so it will bounce, or
        \item it is in a neighbor country $c'$; in such case it is removed from the local individuals and inserted into an outbound list $migrated \mhyphen to_{c'}$ for the specific country.
    \end{enumerate*}.

    \item[exchange migrated] Country $c$ exchanges the list of individuals that moved to each of its neighbors $c'$. The source country performs an \MPIIsend and the destination will perform a matching \MPIRecv , both with the specific tag \MigratedTag.
    We then integrate the received individuals into the local lists and finally wait for the sends to complete.
    We chose a \emph{non-blocking} send so that the process immediately starts receiving from its neighbors, without having to wait for all the outbound data to be copied into the buffers, which may fill since a lot of data is potentially sent. For receives we chose a \emph{blocking} approach, with the disadvantage that a process might be stuck waiting for a slower neighbor, while a faster neighbor has already sent its data. However, the only approaches that overcome this limitation are \emph{event-driven} and \emph{polling}, none of which is available out-of-the-box in MPI, so we chose not to overcomplicate the code for a small gain in performance.
    In order to send the struct representing an individual, we created a custom MPI type called \texttt{mpi\_individual}.

    \item[write summary] If at least one day has passed since the last summary, each country counts how many susceptible, infected and immune individuals are currently in it. This data is packed into a \texttt{mpi\_summary} triple and gathered by the root using \MPIGather; the root will write an entry for each country in a \texttt{csv} file. An alternative to this approach would be collectively writing to an \MPIFile , but this also involves ensuring that each write operation is atomic. In our opinion the chosen approach is simpler, clearer and equivalent in terms of performance.
     
    \item[check termination] Each country counts the local number of infected individuals, then it calls \MPIAllreduce so that the total number of infected in the world is computed and received by each country. If there are no more infected in the whole world, the loop stops.
\end{description}

\begin{algorithm}[p]

    \If{root}{
        \Cfg$\leftarrow \langle N, I, W, L, w, l, v, d, t_{infection}, t_{recovery}, t_{immunity}, t_{step}, t_{target}\rangle$\;
        \ValidateConfig{\Cfg}\;
    }
    \MPIBcast{\Cfg}\;
    \If{root}{
        $\left\{ N_c \right\} \leftarrow$ \DistributePopulation{$N$}\;
        $\left\{ I_c \right\} \leftarrow$ \DistributePopulation{$I$}\;
    }
    \MPIScatter{$N_c$}\;
    \MPIScatter{$I_c$}\;
    \ForEach{individual $i$}{
        $pos_i \leftarrow$ random position\;
        $\Delta pos_i \leftarrow t_{step} \cdot v \cdot
            \begin{bmatrix} \cos \theta_i & \sin \theta_i \end{bmatrix}$,
        where $\theta_i$ is a random direction\;
        $status_i \leftarrow$ \NotExposed | \Infected, based on distribution\;
        $t_{status} \leftarrow 0$\;
    }
    \caption{Initialization}
    \label{alg:initialization}
\end{algorithm}
\begin{algorithm}[p]

    \For{$t \leftarrow 0$ \KwTo $t_{target}$ \Step $t_{step}$}{
        \ForEach{susceptible individual $i$, infected individual $j$}{
            \UpdateExposure{$i,j$}\;
        }
        \ForEach{individual $i$}{
            \UpdateStatus{i}\;
            \UpdatePosition{i}\;
        }
        $\left\{ migrated \mhyphen to_{c'} \right\} \leftarrow$ list of individuals to be moved from $c$ to each neighbor $c'$\;
        \ForEach{neighbor country $c'$}{
            \MPIIsend{$migrated \mhyphen to_{c'}$, $c'$}\;
        }
        \ForEach{neighbor country $c'$} {
            \MPIRecv{$migrated \mhyphen from_{c'}$, $c'$} and integrate it with local individuals\;
        }
        wait until all the \MPIIsend have completed\;
        \If{end of day}{
            $\langle susceptible_c, infected_c, immune_c \rangle \leftarrow$ \CountByStatus{$\left\{ status_{i} \right\}$}\;
            $summary \leftarrow$ \MPIGather{$\left\{\langle susceptible_c, infected_c, immune_c \rangle\right\}$}\;
            \If{root}{
                \WriteToCSV{$summary$}\;
            }
        }
        $infected_c \leftarrow$ \Count{infected individuals}\;
        $total \mhyphen infected \leftarrow$ \MPIAllreduce{$infected_c$}\;
        \If{$total \mhyphen infected = 0$}{
            \Break\;
        }
    }
    \caption{Main Loop}
    \label{alg:mainloop}
\end{algorithm}
